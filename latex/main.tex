% This version of CVPR template is provided by Ming-Ming Cheng.
% Please leave an issue if you found a bug:
% https://github.com/MCG-NKU/CVPR_Template.

%\documentclass[review]{cvpr}
\documentclass[final]{cvpr}

\usepackage{times}
\usepackage{epsfig}
\usepackage{graphicx}
\usepackage{amsmath}
\usepackage{amssymb}

% Include other packages here, before hyperref.

% If you comment hyperref and then uncomment it, you should delete
% egpaper.aux before re-running latex.  (Or just hit 'q' on the first latex
% run, let it finish, and you should be clear).
\usepackage[pagebackref=true,breaklinks=true,colorlinks,bookmarks=false]{hyperref}


\def\cvprPaperID{****} % *** Enter the CVPR Paper ID here
\def\confYear{CVPR 2021}
%\setcounter{page}{4321} % For final version only


\begin{document}

%%%%%%%%% TITLE
\title{\LaTeX\ Author Guidelines for CVPR Proceedings}

\author{Quinn Murphey\\
University of Texas at San Antonio\\
1 UTSA Circle San Antonio, TX\\
{\tt\small quinn.murphey@my.utsa.edu}
% For a paper whose authors are all at the same institution,
% omit the following lines up until the closing ``}''.
% Additional authors and addresses can be added with ``\and'',
% just like the second author.
% To save space, use either the email address or home page, not both
\and
Adrian Ramos\\
University of Texas at San Antonio\\
1 UTSA Circle San Antonio, TX\\
{\tt\small adrian.ramos@my.utsa.edu}

\and
Gabriel Soliz\\
University of Texas at San Antonio\\
1 UTSA Circle San Antonio, TX\\
{\tt\small gabriel.soliz@my.utsa.edu}
}

\maketitle


%%%%%%%%% ABSTRACT
\begin{abstract}
    Generative Adversarial Networks (GAN) have become a precedent for
    unsupervised generative modeling. While remaining infamous for being
    difficult to train and requiring an abundance of focused research. There are
    pre-existing GAN models, which we can use as a basis for our model. We will
    combine the ideology of GAN with Game Theory modifications implemented into
    a basic GAN model. We begin with a basic GAN model and present new
    algorithms to establish Games of GANs. With more research being conducted on
    the topic, there is a stable foundation of information to build off.
    Finally, we discuss the results of our works and attempt to present future
    research possibilities.
\end{abstract}

%%%%%%%%% BODY TEXT

\section{Introduction}

The price fluctuation of good and stocks are often difficult to predict due to
the numerous amounts of variables that play an important role of the price
function.  While there exists research that reflects on those expected
variables, such as that of Romero \cite{romero}, who used a variety of Long
Short-Term Memory (LSTM) models and compared results to that of a GAN. And the
research conducted by Srivastava, Khare and Vidhya \cite{srivastava} which
compares multiple results comprised from other researchers and their unique test
leading to their results.  However, there has been minimal research on the price
fluctuation of goods and stocks due to external events, such as war, pandemics,
or environmental catastrophes. While reports have been brought up that show
certain effects of specific tragedies, such as the COVID-19 pandemic report by
Mead, Ransom, Reed, and Sager \cite{mead}. The rate that prices fluctuate of
goods and stocks during times of crisis and compared to other times of crisis
could potentially help uncover areas which are most impacted. Including
opportunities for potential preventive measures to attempt to thwart a severe
effect.  In order to conduct and produce effective results, we will compare our
results from the data sets produced by Kesternich, Siflinger, Smith, and Winter
\cite{kesternich}, and that from Boysen \cite{boysen}, and also that from
Mouchtaris, Sofianos, Gogas, and Thophilos \cite{mouchtaris}. We will attempt to
measure the expected and actual price fluctuation, per each unique catastrophe
and how it developed over time of the catastrophe.

The source code for our project can be found at 
\url{https://www.github.org/Nragis/cs4263-project}.


\section{Proposed Approach}

For this project we will approach it in our own unique way. We will utilize the
Energy Information Agency's Natural Gas dataset spanning the past several
years. We will also utilize a time-series regression algorithm to analyze and
predict the price for natural gas. Using a time-series regression algorithm
should help us with utilizing and processing the data set we have chosen to its
fullest extent utilizing every bit of knowledge we have to give an accurate
prediction not only of the past but also the future. Utilizing this method our
prediction data should be superior to the traditional econometric models and
have the ability to predict future data points.

\section{Experiments}

\section{Results}

\section{Related Work}

From what we have observed there seems to be certain trends when trying to
predict natural gas prices. The trend majority of the articles such as
“Forecasting Natural Gas Spot Prices with Machine Learning” use is by taking
the price of the gas as far as you have a data set for and then using
adaptive and regression models to predict the gas prices future. The next
theme that some articles use such as “Deep Neural Network Model for
Improving Price Prediction of Natural Gas” is that they look at the current
trend of natural gas and other similar items on something like google and if
there is a trend of natural gas possibly becoming volatile with other
forecasts also coming to this conclusion then it changes the prediction
accordingly. The least common way that I have found is one explored in the
paper “Natural Gas Price Prediction with Big Data” where the authors use
sentiment analysis on a large body of literature, most commonly the news.
This way while uncommon is surprisingly effective with it being able to tell
the sentiment within the text and according to how drastic it is it changes
the predictions.

\nocite{*}

{\small
    \bibliographystyle{ieee_fullname}
    \bibliography{egbib}
}

\end{document}
